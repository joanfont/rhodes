%!TeX root=MemoriaTFG.tex
\chapter{Desenvolupament del projecte}\label{desenvolupament}

En aquest capítol es veurà com s'ha desenvolupat el projecte i es comentaran les tècniques de desenvolupament usades. També es comentarà com s'ha usat el repositori \emph{git} tant per mantenir un control de versions com per fer les pujades a producció.

\section{Desenvolupament del projecte}
Aquest projecte s'ha fet mitjançant mètodes àgils de desenvolupament de software basats en realització iterativa i incremental.\\

Abans d'entrar en qüestió s'ha de comentar que aquest projecte nodreix a dos altres projectes més. Es tracta de dues aplicacions mòbils, una pel sistema operatiu Android i l'altre per la plataforma iOS. La coordinació entre els tres projectes ha estat una peça fonamental del desenvolupament del projecte.\\

Cada nova funcionalitat nova va acompanyada dels seus corresponents \emph{unit tests} per comprovar el seu correcte funcionament i que no alteri el correcte funcionament de les funcionalitats ja desplegades. Si aquests tests han passat satisfactòriament es pot fer la pujada a producció (els tests es passen primer en l'entorn de desenvolupament local). Cal recordar que quan es fa una pujada a producció automàticament es passen els tests mitjançant l'eina \emph{Travis CI}. Si el resultat d'aquesta passada ha estat satisfactòria mantenim els canvis. Del contrari, tornem a una versió anterior on els tests havien passat.\\

Quan s'havia desplegat una nova funcionalitat al servidor de testeig, els programadors dels altres dos projectes ho començaven a integrar dins la seva aplicació i donaven \emph{feedback} al desenvolupador de l'\ac{API} de possibles millores o falles en el sistema.

\section{Git}

Per dur a terme el control de versions i les pujades a producció del projecte s'ha emprat l'eina \emph{git}. El principal motiu per emprar un sistema de control de versions és per la possibilitat de tornar enrere en cas de fallida i mantenir un històric dels canvis que s'han fet a cada fitxer del repositori.\\

\emph{Git} és capaç de funcionar amb diferents línies de desenvolupament, anomenades branques. Hem estructurat el projecte amb dues branques:

\begin{itemize}
	\item \textbf{\emph{master}:} codi testejat i que es troba al servidor de producció.
	\item \textbf{\emph{develop}:} codi actualment en desenvolupament i que no s'ha testejat encara
\end{itemize}

Una vegada acabat el desenvolupament d'una funcionalitat, es testeja i si el resultat és satisfactori s'integra dins la branca \emph{master}.\\

Un altre us que s'ha donat \emph{git} ha estat, amb el suport del servei per allotjar repositoris \emph{online} \emph{GitHub}, una eina de suport pel desplegament de l'aplicació al servidor de testeig.\\

El mateix codi que es té en local, s'ha anat publicant a \emph{GitHub} a mode de còpia de seguretat. Quan s'ha fet un desplegament, s'han descarregat els canvis de \emph{GitHub} al servidor de testeig.

\section{Desplegament}

Quan una nova funcionalitat està llesta i testejada per ser pujada al servidor de testeig o desenvolupament, s'ha emprat l'eina \emph{Fabric}. \\

Aquesta eina ens permet executar accions remotes a un o varis servidors. El desenvolupador ha de codificar l'acció que vol realitzar a un fitxer \texttt{fabfile.py} i situar-lo a l'arrel del projecte. Cada acció està definida per una funció. A continuació es llisten les principals accions del nostre arxiu \texttt{fabfile}.

\begin{itemize}
	\item \textbf{update\_dependencies:} comanda per a actualitzar les dependències de les llibreries. \texttt{fab update\_dependencies} instal·la les noves dependències declarades al fitxer \texttt{requirements.txt}.
	\item \textbf{pull\_branch:} executa la comanda \texttt{git pull} a la carpeta del projecte del servidor remot. Es descarrega els canvis del repositori.
	\item \textbf{restart:} reinicia tant el \emph{gunicorn} com l'\emph{nginx} al servidor remot. Aquesta acció torna a carregar el codi al servidor d'aplicacions \emph{gunicorn}.
	\item \textbf{deploy:} executa les tres comandes anteriors. Aquesta és la comanda usada per fer desplegaments a testeig o a producció, agilitzant la tasca al desenvolupador.
\end{itemize}