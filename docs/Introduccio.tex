%!TeX root=MemoriaTFG.tex
\chapter{Introducció}
En aquest capítol es dona una visió inicial del projecte. A la secció \ref{contexte} es posa en context el projecte. A la secció \ref{estructura_document} s'explica com esta estructurada la memòria del projecte, capítol a capítol.

\section{Contextualització del projecte}\label{contexte}
Actualment, la missatgeria que proporciona l'entorn Moodle de la \ac{UIB}, el Campus Extens, no és del tot instantània. Els missatges s'envien a la mitja hora d'haver-los redactat, fet que provoca que el missatge no arribi als destinataris al moment d'haver-se enviat. \\

El fet de que la notificació d'un missatge rebut al fòrum d'una assignatura sigui instantània cobra més amb temes d'avisos a l'alumnat o al professorat d'aquesta. Possibles retards del professor, canvi de l'aula de classe o un canvi d'horari són clars exemples de missatges que es vol que arribin de manera immediata.\\

Quan usam el telèfon mòbil per consultar el correu, no usam el servei de \emph{webmail} que ens pugui proporcionar el nostre proveïdor de correu electrònic. Quan l'usam per consultar una xarxa social no hi accedim pel seu web si no que usam l'aplicació que ells han desenvolupat. 

 
\section{Estructura del document}\label{estructura_document}
La memòria del projecte consta de sis capítols i dos annexes. A part del capítol introductori del projecte, la resta de capítols reflecteixen el treball realitzat.\\

Al capítol \ref{entorn} es fa un repàs de les tecnologies, estàndards i tècniques de desenvolupament per a solucions de missatgeria. S'introduiran conceptes com \ac{SOA}, \ac{REST}, \ac{ORM} i es revisaran d'altres com \ac{HTTP} i \ac{API}.\\

Al capítol \ref{analisi} hi podem trobar l'anàlisi del problema a solucionar. En primer lloc es descriuen els objectius del projecte, quin problema ha de solucionar. En segon lloc es defineixen els requisits tant a nivell d'usuari com a nivell de sistema que el producte final ha de satisfer. Per finalitzar es defineixen els requisits tecnològics del projecte i les solucions \emph{software} proposades.

Al capítol \ref{disseny} s'explica detalladament la solució al problema. Es comença des de una visió global de l'aplicació que s'ha dissenyat. A continuació es descriu el model de dades de l'aplicació. Seguidament s'exposa l'arquitectura de serveis dissenyada juntament amb els serveis desenvolupats. Després s'explica com s'ha desenvolupat l'\ac{API} \ac{REST} i els seus principals components. Tot seguit 

Al capítol \ref{desenvolupament} s'explica com s'ha duit a terme el desenvolupament del projecte. Es comentaran les tècniques de desenvolupament de software i s'explicarà com s'ha usat el repositori \emph{git}.\\

Al capítol \ref{conclusions} es presenten els resultats finals del projecte, una reflexió personal de l'autor i possibles millores o futures funcionalitats del servei. \\

A l'annex \ref{instalacio} es detalla el procediment a seguir instalar tot l'entorn, desde la definició dels requisits previs que el servidor necessita per poder executar l'aplicació, un manual pas per pas d'instalació del servidor i les pautes per configurar el sistema pel seu correcte funcionament.\\

A l'annex \ref{manual} es descriu de manera detallada cada acció de la API dissenyada per a que es pugui extendre o desenvolupar nous clients que s'alimentin del servei. \\

