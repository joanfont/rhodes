%!TeX root=MemoriaTFG.tex
\chapter{Introduccio}
En aquest capítol es dona una visió inicial sobre el projecte. A la secció \ref{contexte} es posa en contexte el projecte. En la secció \ref{motivacio} s'explica perquè s'ha decidit dur endevant aquest projecte. Per finalitzar, a la secció \ref{estructura_document} s'explica com esta estructurada la memoria del projecte, capítol a capítol.
\section{Contextualització del projecte}\label{contexte}
Actualment, la missatgeria que proporciona l'entorn Moodle de la universitat, el Campus Extens, no és del tot instantània. Els missatges s'envien a la mitja hora d'haver-los redactat, fet que provoca que el missatge no arribi als destinataris al moment d'haver-lo enviat. \\

El fet de que la notificació d'un missatge rebut al fòrum d'una assignatura sigui instantania cobra més amb temes d'avisos a l'alumnat o al professorat d'aquesta. Possibles retrassos del professor, canvi de l'aula de classe o un canvi d'horari són clars exemples de missatges que es vol que arribin de manera inmediata.\\

Quan usam el telèfon mòbil per consultar el correu, no usam el servei de webmail que ens pugui proporcionar el nostre proveidor de correu electrònic. Quan l'usam per consultar una xarxa social no hi accedim pel seu web si no que empleam l'aplicació que ells han desenvolupat. 

\section{Motivació}\label{motivacio}
 % TODO
 
\section{Estructura del document}\label{estructura_document}
La memoria del projecta consta de cinc capítols i dos annexes. A part del capítol introductori del projecte, la resta de capítols reflexen el treball realitzat.\\

Al  capítol \ref{entorn} es fa un anàlisi de les solucions de serveis de missatgeria que podem trobar actualment, de quin és l'estat actual dels serveis de missatgeria.\\

Al capítol \ref{analisi} hi podem trobar l'anàlisi del problema a solucionar. En primer lloc es descriuen els objectius del projecte, quin problema ha de solucionar. En segon lloc es defineixen els requisits tant a nivell d'usuari com a nivell de sistema que el producte final ha de satisfer.\\

Al capítol \ref{disseny} s'explica detalladament la solució al problema. Es comença desde una visió global de l'aplicació que s'ha dissenyat, es descriuen els dos prinipals mòduls de l'aplicació i les seves principals funcionalitats. Posteriorment es fa una descripció en més detall de cada un dels mòduls.\\

Al capítol \ref{conclusions} es presenten els resultats finals del projecte, una reflexió personal de l'autor i possibles millores o futures funcionalitats del servei. \\

A l'annex \ref{importacio} es pot trobar una guia detallada de com importar les dades de qualsevol universitat a l'aplicació desenvolupada.

A l'annex \ref{instalacio} es detalla el procediment a seguir instalar tot l'entorn, desde la definició dels requisits previs que el servidor necessita per poder executar l'aplicació, un manual pas per pas d'instalació del servidor i les pautes per configurar el sistema pel seu correcte funcionament.\\

A l'annex \ref{manual} es descriu de manera detallada cada acció de la API dissenyada per a que es pugui extendre o desenvolupar nous clients que s'alimentin del servei. \\

