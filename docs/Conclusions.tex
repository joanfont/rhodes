% !TEX root=MemoriaTFG.tex

\chapter{Conclusions}\label{conclusions}

En aquest darrer capítol es reflexiona i es conclou el projecte. En primer lloc veurem els resultats obtinguts i els principals problemes trobats durant l'execució del projecte.\\

Seguidament es veuran futures millores al projecte com funcionalitats noves que es puguin afegir al sistema. Per finalitzar es veurà l'opinió personal de l'autor del projecte.\\

\section{Resultats}

El propòsit del projecte d'aquest \ac{TFG} era dissenyar i implementar un sistema de missatgeria per a una comunitat universitària. Aquest servei està pensat per que després es puguin nodrir d'ell diversos clients (mòbils, web o d'escriptori).\\

Per això, s'ha dissenyat i implementat una \ac{API} \ac{REST} per donar suport a la missatgeria. Aquesta \ac{API} contempla dos tipus d'usuaris: els professors i els alumnes. Permet enviar missatges a les assignatures i als grups que l'usuari pertany. També es possibilita l'enviament de missatges personals entre professors i alumnes. Als missatges enviats se li poden adjuntar fitxers.\\

A més, s'ha eliminat tota dependència amb l'entorn tecnològic de la universitat. El sistema manté una còpia de les dades necessàries per operar i no manté contacte amb els sistemes propis de la universitat, fet que possibilita que aquesta \ac{API} pugui ser explotada a qualsevol universitat o institució educativa.\\

\section{Futures millores}

En primer lloc es podria implementar el nou format de sortida de les dades de l'\ac{API}. Només implementant un nou \emph{mixin} com el \texttt{JSONResponseMixin} que donés suport a la serialització de les dades en format \ac{XML}. L'usuari podria especificar en quin format vol les dades amb un paràmetre al \emph{query string} de la petició. \\

Es podria realitzar una millora en la gestió de  l'emmagatzemament multimèdia portant-lo a un altre servidor o delegar-ho a un servei   \emph{cloud}. Aquesta millora permetria alliberar espai al disc i la càrrega del servidor de producció. Com s'ha comentat a la seccio \ref{contingut_multimedia} només faria falta re-escriure una nova classe com la de \texttt{DiskStorage} que implementi els mateixos mètodes però per la nova destinació dels fitxers.\\

S'ha comentat en nombroses ocasions que el sistema no està arrelat a les dades de la universitat. Es podrien crear fàcilment altres usuaris o inclús altres tipus d'usuari. La idea seria crear un usuari super-administrador que pogués enviar missatges a qualsevol assignatura, grup o usuari registrat al sistema. Aquest nou usuari podria actuar a mode de avisos per a tota la comunitat universitària o per uns estudis en general. També es podrien crear els rols de cap d'estudis, director, degà que tingués permisos per enviar missatges a un espectre d'usuaris més ampli.


\section{Opinió personal}

El desenvolupament del projecte m'ha suposat una experiència molt enriquidora. Tant des de el punt de vista tècnic com del punt de vista de organitzatiu.\\

Per la banda tècnica, ha estat la primera \ac{API} \ac{REST} que he desenvolupat per complet, des de zero i pràcticament fora cap \emph{framework}, com Django Rest Framework\cite{djangorestframework},  que doni suport al \ac{REST}. La decisió de fer-ho fora cap \emph{framework} ha estat per poder tenir més control sobre el que es fa en cada moment i de desenvolupar un sistema totalment desacoblat del \emph{framework} que atén i respon peticions web. He de reconèixer que aquesta decisió ha derivat en bastants problemes i que hi he dedicat més temps del que hagués pogut dedicar-li si hagues empleat una solució com Django Rest Framework. Però gràcies a aquests entrebancs he pogut aprendre nous conceptes que fins al moment m'eren desconeguts.\\

Cal dir que abans de realitzar el projecte, ja havia fet algunes aplicacions web amb \emph{Flask} però no al mateix nivell que ho he fet en aquest projecte. He pogut conèixer en detall aquest petit \emph{framework} i he pogut comprovar lo senzill que és per dintre. \\


Per la part organitzativa, he viscut en primera persona el que és desenvolupar software per a usuaris que no són jo mateix. Els dos alumnes encarregats dels dos clients mòbils han donat la seva opinió, proposat noves funcionalitats i comunicat possibles millores en el sistema. Part d'aquestes millores han estat acceptades, sobretot les que fan referència al format de sortida de les dades

