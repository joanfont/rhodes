% !TEX root=MemoriaTFG.tex

\chapter{Resum executiu}
El propòsit d'aquest projecte és desenvolupar una \ac{API} per donar suport a la missatgeria de la comunitat universitària. Aquest missatgeria ha de facilitar l'enviament de missatges a fòrums per cada assignatura, fòrums per cada grup d'assignatura i missatges directes entre integrants de la comunitat universitària. Aquests missatges poden dur fitxers adjuntats. \\

Les dades amb les que opera aquesta \ac{API} no són les mateixes dades amb les que la universitat opera. És manté una còpia de les dades al servidor de l'aplicació per no tractar directament amb els sistemes de la universitat.\\

L'\ac{API} desenvolupada opera sobre el protocol \ac{HTTP}, segueix els estàndards \ac{REST}, usa un \ac{ORM} com a eina per a accedir a la base de dades i retorna les dades en format \ac{JSON}. \\

Es poden distingir dos principals mòduls al producte final. Un és el mòdul que conté els serveis de l'aplicació. Aquests serveis s'han dissenyat seguint l'arquitectura \ac{SOA}. \\

L'altre mòdul és l'\ac{API} \ac{REST} que s'encarrega d'atendre peticions. El mòdul de serveis és independent de l'\ac{API} i es pot usar per exemple per desenvolupar una aplicació d'escriptori.\\

Per assegurar que cada nova funcionalitat desenvolupada no altera el correcte funcionament de les ja existents, s'ha elaborat una bateria de proves unitàries que validen el correcte funcionament del sistema cada vegada que es fa un desplegament.




