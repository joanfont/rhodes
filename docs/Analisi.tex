%!TeX root=MemoriaTFG.tex

\chapter{Anàlisi del problema}\label{analisi}
%En aquest capítol trobam un anàlisi del problema a solucionar. Es descriuen els objectius del projecte, quins requisits tant a nivell d'usuari com de sistema ha de satisfer el producte final. A continuació s'explicaràn les tecnologies usades pel desenvolupament de la \ac{API}.

\section{Objectius}
El principal objectiu d'aquest projecte és desenvolupar una \ac{API} \ac{REST} per donar suport a la missatgeria entre integrants de la comunitat universitària. Aquesta \ac{API} ha de ser totalment independent dels sistemes propis de la universitat, motiu pel qual es desenvoluparà un sistema totalment aïllat de les dades de la universitat. \\

Els objectius generals del projecte són:
\begin{itemize}
	\item Estudiar i analitzar el sistema actual que s'usa dins l'àmbit universitari per a la comunicació entre docent i alumne.
	\item Dissenyar i desenvolupar una \ac{API} per donar suport a la missatgeria entre integrants de la comunitat universitaria.
	\item Proporcionar un servei per que pugui ser explotat desde diversos clients (aplicacions mòbils, aplicacions web, aplicacions d'escriptori, etc...)
	\item Aïllar el sistema de les dades de la universitat per que aquest pugui ser portable a qualsevol altre universitat o institució educativa.
\end{itemize}
\section{Requisits d'usuari}
Hem dividit els requisits per alumnes i professors. En primer lloc veurem els requisits per l'alumne, en segon lloc veurem els requisits pel professor.
\subsection{Requisits d'usuari a nivell d'alumne}
Els requisits d'usuari a nivell d'alumne són els següents:
\begin{itemize}
	\item Un alumne podrà entrar al sistema amb les seves credencials de la universitat.
	\item Un alumne podrà enviar un missatge al fòrum de cada assignatura on està matriculat.
	\item Un alumne podrà llegir tots els missatges del fòrum de cada assignatura on està matriculat.
	\item Un alumne podrà enviar un missatge al fòrum del seu grup de cada assignatura on està matriculat.
	\item Un alumne podrà llegir tots els missatges del fòrum del seu grup de cada assignatura on està matriculat.
	\item Un alumne podrà veure els professors de cada assignatura on està matriculat.
	\item Un alumne podrà enviar un missatge a cada professor que imparteix una assignatura on ell està matriculat.
	\item Un alumne podrà llegir tots els missatges que cada professor que imparteix una assignatura on ell està matriculat li ha enviat.
	\item Un alumne podrà canviar la seva fotografia de perfil.
\end{itemize}

\subsection{Requsits d'usuari a nivell de professor}
Els requisits d'usuari a nivell de professor són els següents:
\begin{itemize}
	\item Un professor podrà entrar al sistema amb les seves credencials de la universitat.
	\item Un professor podrà enviar un missatge al fòrum de cada assignatura que imparteix.
	\item Un professor podrà llegir tots els missatges del fòrum de cada assignatura que imparteix.
	\item Un professor podrà enviar un missatge al fòrum de cada grup de les assignatures que ell imparteix
	\item Un professor podrà llegir tots els missatges del fòrum de cada grup de les assignatures que ell imparteix
	\item Un professor podra veure els professors de cada assignatura que imparteix.
	\item Un professor podrà veure els alumnes matriculats de cada assignatura que imparteix.
	\item Un professor podrà enviar un missatge a cada professor que imparteix una assignatura que ell imparteix.
	\item Un professor podra llegir tots els missatges que cada professor que imparteix una assignatura que ell imparteix li ha enviat.
	\item Un professor podrà enviar un missatge a cada alumne matriculat a les assignatures que ell imparteix.
	\item Un professor podra llegir tots els missatges que cada alumne matriculat a les assignatures que ell imparteix li ha enviat.
	\item Un professor podrà canviar la seva fotografia de perfil.
\end{itemize}
\section{Requisits de sistema}
A continuació s'enumeraràn els requisits de sistema del projecte:
\begin{itemize}
	\item El sistema ha de ser accessible per a qualsevol persona de la comunitat universitària que vulgui desenvolupar un client que operi sobre ell.
	\item La connexió client-servidor ha d'estar encriptada baix el protocol \ac{TLS}
	\item El sistema ha de proporcionar informació detallada sobre els errors que hi pugui haver a la petició.
	\item El sistema ha de implementar mètodes de seguretat per assegurar que un usuari només té accés a les dades que està autoritzat a veure.
	\item El sistema ha de poder distingir usuaris entre els diferents rols de la universitat (estudiant, \ac{PAS}, \ac{PDI}, etc...)
\end{itemize}

\section{Anàlisi tecnològic}
En aquesta secció es presenta un anàlisi de les tecnologies que necessitam per a desenvolupar la \ac{API} desitjada. En primer lloc veurem quins requisits han de satisfer aquestes tecnologies. En segon lloc veurem quines tecnologies s'han elegit per dur a terme el desenvolupament.

\subsection{Requisits tecnològics}
A continuació s'exposaràn quins requisits han de tenir els diferents components que composaràn la nostra \ac{API}. 
	
	\subsubsection{Llenguatge de programació}
	El llenguatge de programació usat pel desenvolupament ha de tenir suport per atendre peticions \ac{HTTP}. 
	
	\subsubsection{\ac{SGBD}}
	Per tal d'aconseguir la ja anomenada independència de les dades de la universitat, s'emmagatzemarà una copia d'aquestes al nostre servidor. Aquesta base de dades ha de ser relacional. El \ac{SGBD} que s'usi ha de ser compatible amb l'\ac{ORM} que emprarem.

	
	\subsubsection{\emph{Framework} web}
	El \emph{framework} web que s'utilitzarà ha de proporcionar una manera àgil i senzilla de construïr una \ac{API} \ac{REST}. No té perquè ser un \emph{framework} \emph{full stack} \cite{web_framework}. Ja que el nostre aplicatiu final no requereix de planes en format \ac{HTML}, el motor de renderitzat de plantilles no ha de ser un punt fort d'aquest \emph{framework}.
	
	\subsubsection{Presentació de les dades}
	El format en que la \ac{API} retorna les dades ha de ser senzill, universal i entenible per la majoria de llenguatges de programació.

	\subsubsection{Integració contínua}
	El llenguatge de programació ha d'incorporar tècniques d'\emph{unit testing} per tal d'agilitzar el procés d'integració de noves funcionalitats al producte. També es precisa d'un sistema per assegurar que cada desplegament al servidor de producció o testeig es fa de manera correcta i no altera el funcionament del servei.

\subsection{Tecnologies usades}
Una cop definits els requisits que les tecnologies que es necessiten han de satisfer, es veurà quina és la tecnologia que millor encaixa amb aquests requisits.

	\subsubsection{Llenguatge de programació}
	El llenguatge de programació que s'ha triat per desenvolupar el projecte ha estat \textbf{Python} \cite{python}, per la quantitat de llibreries existents que hi ha, la seva senzillesa, la gran comunitat de desenvolupadors que hi ha darrera i pel fet de que és un llenguatge prou consolidat en l'àmbit web degut al elevat nombre de frameworks web que estan implementats baix aquest llenguatge (Django, Flask, Pyramid, Bottle, WebPy, CherryPy, etc...)

	\subsubsection{\ac{SGBD}}
	El \ac{SGBD} que s'ha triat per desenvolupar el projecte ha estat \textbf{MySQL} \cite{mysql}. S'ha triat aquest \ac{SGBD} per la seva facilitat d'instalaciói configuració i per la familiaritat que té el desenvolupador en desenvolupar solucions sobre aquest \ac{SGBD}.

	\subsubsection{\emph{Framework} web}
	El \emph{framework} web que s'ha triat ha estat \textbf{Flask} \cite{flask} per la seva senzillesa i extensibilitat. Aquest framework es sol usar per a construïr aplicacions lleugeres. El seu motor de renderitzat de plantilles (Jinja \cite{jinja}) és bastant lleuger i no provoca sobre-càrrega al sistema. 

	\subsubsection{Presentació de les dades}
	El format triat en que es presentaran les dades de l'\ac{API} ha estat \ac{JSON} \cite{json}. \ac{JSON} és un estàndard obert basat en text dissenyat per a intercanvi de dades llegible per humans. Deriva del llenguatge JavaScript i es usat per a representar estructures de dades simples i llistes associatives, anomenades objectes. S'utilitza principalment per transmetre dades entre una aplicació de servidor web i, com una alternativa a \ac{XML}.

	\subsubsection{Llibreries}
	Per donar suport al desenvolupament s'han empleat una sèrie de lliberies, a continuació es fa una enumeració d'aquestes.\\

	Com s'ha comentat anteriorment, per donar suport a l'abstracció de les dades de la base de dades, s'ha empleat la tècnica \ac{ORM}. L'\ac{ORM} que s'ha triat pel projecte, desenvolupat també en Python, ha estat \textbf{SQLAlchemy} \cite{sqlalchemy}. SQLAlchemy proporciona la capa d'abstracció que es cerca dins un \ac{ORM}, definició dels models de dades amb les estructures de dades natives del llenguatge (classes).\\

	Per donar suport al desplegament al servidor de testeig o producció s'ha triat la llibreria \textbf{Fabric} \cite{fabric} que ens permet realitzar accions remotes a diferents servidors via \ac{SSH}. D'aquesta manera, amb una simple instrucció a la consola podem desplegar el codi al nostre servidor (o servidors) de manera ràpida i centralitzada.\\

	Per gestionar els canvis a la base de dades sense haver de desenvolupar el codi \ac{SQL} s'ha triat la llibreria de Python \textbf{Alembic} \cite{alembic} que proporciona uns fitxers anomenats migracions que reflexen els canvis a la base de dades que volem realitzar.\\

	Quan parlam d'una aplicació web, és dificil no parlar de concurrència. Cada petició web és un procés del sistema de la nostra aplicació i una part crítica d'aquesta concurrència són les connexions a la base de dades. Flask i SQLAlchemy tenen una extensió, \textbf{Flask-SQLAlchemy} \cite{flask_sqlalchemy}, per donar suport a múltipes connexions a la base de dades sense preocupar-se de la concurrència. \\

	\textbf{Flask-Script} \cite{flask_script} és una eina per invocar scripts de Python que interactuïn amb l'ecosistema de Flask. Podem llançar tests, executar scripts que manipulin dades de la base de dades, etc...\\
	

	\subsubsection{Servidor d'aplicacions i servidor web}
	
	Per servir la nostra aplicació web s'ha triat el servidor d'aplicacions de Python \textbf{gunicorn} \cite{gunicorn} que ens permet d'una manera senzilla configurar-lo i establir-ne el nombre de processos que volem que tengui, la direcció on volem que escolti les peticions o l'usuari que executarà l'aplicació, entre d'altres.\\

	Per no emplear gunicorn com a frontal web, s'ha montat un servidor web, \textbf{nginx} \cite{nginx}, que actua com a proxy invers. Nginx redirigeix tot el tràfic que li arriba cap al servidor d'aplicacions, gunicorn. D'aquesta manera podem gaudir dels avantatges d'nginx, com per exemple: facilitat d'implementar un \emph{load balancer}, implementar redireccions o usar \ac{HTTPS}

	\subsubsection{Control de versions}
	Per mantenir un control de versions del codi del projecte s'ha usat \emph{git} \cite{git} amb \emph{GitHub} \cite{github} com a plataforma online per allotjar-lo. El fet d'usar GitHub ha facilitat el desplegament àgil al servidor de producció.
	

	\subsubsection{Utilitats}

	Per encendre, apagar o reiniciar la nostra aplicació s'ha usat la utilitat de linux \textbf{Supervisor} \cite{supervisor} que proporciona una interfície similars als serveis de Unix per controlar la seva execució. \\

	Per mantenir el nostre codi aïllat de les lliberies de Python que ja té el sistema s'han empleat entorns virtuals. \textbf{virtualenv} \cite{virtualenv} i \textbf{virtualenvwrapper} \cite{virtualenvwrapper} ens ajuden a manejar aquests entorns virtuals. D'aquesta manera podem tenir el nostre projecte en un contexte únic, sense que altres aplicacions residents al mateix servidor puguin alterar, per exemple, les variables d'entorn. Una altre avantatge d'usar entorns virtuals, es que tendrem el control sobre les lliberies que tenim instalades mitjançant \emph{pip} o \emph{easy\_install}.\\

	\subsubsection{Integració contínua}

	Per donar suport a la ja anomenada integració contínua s'ha triat el servei en línia \textbf{Travis CI}. \cite{travis_ci} Aquest servei ofereix la execució de \emph{unit-test} i notificicació dels resultats via \emph{e-mail}. D'aquesta manera quan es fa un desplegament al servidor es reb una notificació del resultat dels tests executats.\\
	
	\textbf{Nosetests} \cite{nose} és una utilitat de Python per executar \emph{unit test} de manera massiva. A un directori del nostre projecte tenim tots els tests que comproven que el nostre sistema funciona correctament. Amb aquesta utilitat només executant la comanda \texttt{nosetests api/tests/*.py} tendrem el resultat de l'execució de tots els \emph{unit test} dins el directori \texttt{api/tests}.\\
	
	La taula \ref{table:tecnologies} mostra una visió global de les tecnologies empleades al projecte.

\begin{table}[here]
 	\begin{center}
 		\begin{tabularx}{\textwidth}{|l|X|}
  			\hline
 			\bfseries Categoria & \bfseries Tecnologia (nom comercial) \\ \hline
			Llenguatge de programació &  Python \\ \hline
			\ac{SGBD} & MySQL \\ \hline
			Framework web & Flask\\ \hline
			\multirow{5}{*}{Llibreries} & SQLAlchemy  \\
 & Fabric \\
 & Alembic \\
 & Flask-SQLAlchemy \\
 & Flask-Script\\ \hline
 Servidor d'aplicacions & gunicorn \\ \hline
 Servidor web & nginx \\ \hline
\multirow{2}{*}{Control de versions} & Git \\ & GitHub \\ \hline
\multirow{3}{*}{Utilitats} & Supervisor \\ & virtualenv \\ & virtualenvwrapper\\ \hline
\multirow{2}{*}{Integració contínua} & nosetests \\ & Travis CI \\ \hline
		\end{tabularx}
	\end{center}
	\label{table:tecnologies}
	\caption{Relació de tecnologies usades al projecte} 
\end{table}